
%%%%%%%%%%%%%%%%%%%%%%%%%%%%%%%%%%%%%%%%%
% University/School Laboratory Report
% LaTeX Template
% Version 3.1 (25/3/14)
%
% This template has been downloaded from:
% http://www.LaTeXTemplates.com
%
% Original author:
% Linux and Unix Users Group at Virginia Tech Wiki 
% (https://vtluug.org/wiki/Example_LaTeX_chem_lab_report)
%
% License:
% CC BY-NC-SA 3.0 (http://creativecommons.org/licenses/by-nc-sa/3.0/)
%
%%%%%%%%%%%%%%%%%%%%%%%%%%%%%%%%%%%%%%%%%

%----------------------------------------------------------------------------------------
%	PACKAGES AND DOCUMENT CONFIGURATIONS
%----------------------------------------------------------------------------------------

\documentclass{article}
\usepackage[margin=1.25in]{geometry}
\usepackage{hyperref}
\usepackage[version=3]{mhchem} % Package for chemical equation typesetting
\usepackage{siunitx} % Provides the \SI{}{} and \si{} command for typesetting SI units
\usepackage{graphicx} % Required for the inclusion of images
\usepackage{natbib} % Required to change bibliography style to APA
\usepackage{amsmath} % Required for some math elements 

\setlength\parindent{1em} % Removes all indentation from paragraphs
\setlength{\parskip}{1em}
\renewcommand{\labelenumi}{\alph{enumi}.} % Make numbering in the enumerate environment by letter rather than number (e.g. section 6)

%\usepackage{times} % Uncomment to use the Times New Roman font

%----------------------------------------------------------------------------------------
%	DOCUMENT INFORMATION
%----------------------------------------------------------------------------------------

\title{Check-in} % Title

\author{Weixiong Zheng} % Author name

\date{\today} % Date for the report

\begin{document}

\maketitle % Insert the title, author and date
% If you wish to include an abstract, uncomment the lines below
% \begin{abstract}
% Abstract text
% \end{abstract}

%----------------------------------------------------------------------------------------
%	SECTION 0
%----------------------------------------------------------------------------------------
\section{Updates on previous goals}
Last week's goal was to move on restoration while prepare things for Alex.
%----------------------------------------------------------------------------------------
%	SECTION 1
%----------------------------------------------------------------------------------------
\section{Progress up to now}
Other than the ordinary part of moving on.

\subsection{Preparation for Alex}
The task I assigned to Alex was not simply moving things from devel to restart. The reason
is that through rethinking, I redesigned the data structure of {\tt MaterialProperties} class.
To get Alex started, I prepared all the necessary data-structure change in header so that
Alex can do the moving work in source file with the functions in devel and data structures
I put in header file.

\subsection{Simplification of data input}
Original design of class and functions asks for a lot of arguments. For instance, the
{\tt EquationBase} constructor is declared as:
\begin{verbatim}
template<int dim>
EquationBase<dim>::EquationBase (dealii::ParameterHandler &prm, 
    std::unique_ptr<AQBase<dim>> &aq_ptr,
    std::unique_ptr<MeshGenerator<dim>> &msh_ptr,
    std::unique_ptr<MaterialProperties> &mat_ptr);
\end{verbatim}

This potentially causes issues if developers do not use cautions. So a improvement that
has been done during the past weekend was putting things together using a {\tt struct}
\begin{verbatim}
template<int dim>
struct ComputingData {
  std::unique_ptr<AQBase<dim>> &aq_ptr;
  std::unique_ptr<MeshGenerator<dim>> &msh_ptr;
  std::unique_ptr<MaterialProperties> &mat_ptr;
};
\end{verbatim}
s.t. once {\tt EquationBase} is implemented, it will be much simpler and more readable
\begin{verbatim}
template<int dim>
EquationBase<dim>::EquationBase (dealii::ParameterHandler &prm, 
    std::unique_ptr<ComputingData<dim>> &data_ptr);
\end{verbatim}

The same idea will be applied to containing data for functions related to iterations. 
For instance, there will be a {\tt struct} called {\tt IterationData<dim>} with declaration of
\begin{verbatim}
template <int dim>
struct IterationData {
  std::vector<dealii::Vector<double>> &sflxes_proc;
  std::vector<std::unique_ptr<EquationBase<dim>>> &equ_ptrs;
  std::unique_ptr<IGBase<dim>> &ig_ptr;
  std::unique_ptr<MGBase<dim>> &mg_ptr;
  std::unique_ptr<EigenBase<dim>> &eig_ptr;
}
\end{verbatim}
The goodness we are gonna see at the end is the simplification of various function definitions.
For instance, the {\tt Iterations<dim>::solve\_problems} is defined as
\begin{verbatim}
template <int dim>
void Iterations<dim>::solve_problems (std::vector<Vector<double>> &sflxes_proc,
    std::vector<std_cxx11::shared_ptr<EquationBase<dim> > > &equ_ptrs,
    std_cxx11::shared_ptr<IGBase<dim> > ig_ptr,
    std_cxx11::shared_ptr<MGBase<dim> > mg_ptr,
    std_cxx11::shared_ptr<EigenBase<dim> > eig_ptr) {...}
\end{verbatim}
After implementing the proposed struct, we are having a much simpler function argument
set
\begin{verbatim}
template <int dim>
void Iterations<dim>::solve_problems (
    std::unique_ptr<IterationData<dim>> &itr_data_ptr) {...}
\end{verbatim}
Hooray!
%----------------------------------------------------------------------------------------
%	SECTION 2
%----------------------------------------------------------------------------------------
\section{Things you need from Rachel}


%----------------------------------------------------------------------------------------
%	SECTION 3
%----------------------------------------------------------------------------------------
\section{Goals/Things will be going on}
I will continue whenever I have time this week as this is a moving week. Also I will see if
Alex asks for help.

The other important thing is to finish up the first draft of introduction slides with equations.

%----------------------------------------------------------------------------------------
%	SECTION 4
%----------------------------------------------------------------------------------------
%\section{Links to any related materials}



%----------------------------------------------------------------------------------------
%	BIBLIOGRAPHY
%----------------------------------------------------------------------------------------

%\bibliographystyle{apalike}
%
%\bibliography{sample}

%----------------------------------------------------------------------------------------


\end{document}