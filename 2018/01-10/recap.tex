%%%%%%%%%%%%%%%%%%%%%%%%%%%%%%%%%%%%%%%%%
% University/School Laboratory Report
% LaTeX Template
% Version 3.1 (25/3/14)
%
% This template has been downloaded from:
% http://www.LaTeXTemplates.com
%
% Original author:
% Linux and Unix Users Group at Virginia Tech Wiki 
% (https://vtluug.org/wiki/Example_LaTeX_chem_lab_report)
%
% License:
% CC BY-NC-SA 3.0 (http://creativecommons.org/licenses/by-nc-sa/3.0/)
%
%%%%%%%%%%%%%%%%%%%%%%%%%%%%%%%%%%%%%%%%%

%----------------------------------------------------------------------------------------
%	PACKAGES AND DOCUMENT CONFIGURATIONS
%----------------------------------------------------------------------------------------

\documentclass{article}
\usepackage[margin=0.5in]{geometry}
\usepackage{hyperref}
\usepackage[version=3]{mhchem} % Package for chemical equation typesetting
\usepackage{siunitx} % Provides the \SI{}{} and \si{} command for typesetting SI units
\usepackage{graphicx} % Required for the inclusion of images
\usepackage{natbib} % Required to change bibliography style to APA
\usepackage{amsmath} % Required for some math elements 

\setlength\parindent{1em} % Removes all indentation from paragraphs
\setlength{\parskip}{1em}
\renewcommand{\labelenumi}{\alph{enumi}.} % Make numbering in the enumerate environment by letter rather than number (e.g. section 6)

%\usepackage{times} % Uncomment to use the Times New Roman font

%----------------------------------------------------------------------------------------
%	DOCUMENT INFORMATION
%----------------------------------------------------------------------------------------

\title{Weekly Recap} % Title

\author{Weixiong Zheng} % Author name

\date{\today} % Date for the report

\begin{document}

\maketitle % Insert the title, author and date
% If you wish to include an abstract, uncomment the lines below
% \begin{abstract}
% Abstract text
% \end{abstract}

%----------------------------------------------------------------------------------------
%	SECTION 0
%----------------------------------------------------------------------------------------
\section{Updates on previous goals}
Previous goals was to make google test work and discuss the unit test with Josh.

%----------------------------------------------------------------------------------------
%	SECTION 1
%----------------------------------------------------------------------------------------
\section{Progress up to now}
The fact is when trying to tweak CMake to work with google test, met several difficulty. 
In comparison, the resources on CTest are available on deal.II's website for instructions 
on personal project test suites. After the conversation I had with Josh, we agreed on 
switching to CTest.

I looked through the CTest before the conversation and here are the potential benefits.
\begin{itemize}
	\item Examples available on deal.II website.
	\item Simplicity of setting up CMakeList.txt.
	\item Compatibility with MPI.
\end{itemize}

%----------------------------------------------------------------------------------------
%	SECTION 2
%----------------------------------------------------------------------------------------
\section{Things you need from Rachel}
Not up to now.

%----------------------------------------------------------------------------------------
%	SECTION 3
%----------------------------------------------------------------------------------------
\section{Goals for the coming week}
Up to the point of group meeting, a demo shall be showed on how CTest works with BART.

%----------------------------------------------------------------------------------------
%	SECTION 4
%----------------------------------------------------------------------------------------
%\section{Links to any related materials}



%----------------------------------------------------------------------------------------
%	BIBLIOGRAPHY
%----------------------------------------------------------------------------------------

%\bibliographystyle{apalike}
%
%\bibliography{sample}

%----------------------------------------------------------------------------------------


\end{document}