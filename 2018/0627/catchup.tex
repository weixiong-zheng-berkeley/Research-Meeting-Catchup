
%%%%%%%%%%%%%%%%%%%%%%%%%%%%%%%%%%%%%%%%%
% University/School Laboratory Report
% LaTeX Template
% Version 3.1 (25/3/14)
%
% This template has been downloaded from:
% http://www.LaTeXTemplates.com
%
% Original author:
% Linux and Unix Users Group at Virginia Tech Wiki 
% (https://vtluug.org/wiki/Example_LaTeX_chem_lab_report)
%
% License:
% CC BY-NC-SA 3.0 (http://creativecommons.org/licenses/by-nc-sa/3.0/)
%
%%%%%%%%%%%%%%%%%%%%%%%%%%%%%%%%%%%%%%%%%

%----------------------------------------------------------------------------------------
%	PACKAGES AND DOCUMENT CONFIGURATIONS
%----------------------------------------------------------------------------------------

\documentclass{article}
\usepackage[margin=1.25in]{geometry}
\usepackage{hyperref}
\usepackage[version=3]{mhchem} % Package for chemical equation typesetting
\usepackage{siunitx} % Provides the \SI{}{} and \si{} command for typesetting SI units
\usepackage{graphicx} % Required for the inclusion of images
\usepackage{natbib} % Required to change bibliography style to APA
\usepackage{amsmath} % Required for some math elements 

\setlength\parindent{1em} % Removes all indentation from paragraphs
\setlength{\parskip}{1em}
\renewcommand{\labelenumi}{\alph{enumi}.} % Make numbering in the enumerate environment by letter rather than number (e.g. section 6)

%\usepackage{times} % Uncomment to use the Times New Roman font

%----------------------------------------------------------------------------------------
%	DOCUMENT INFORMATION
%----------------------------------------------------------------------------------------

\title{Check-in} % Title

\author{Weixiong Zheng} % Author name

\date{\today} % Date for the report

\begin{document}

\maketitle % Insert the title, author and date
% If you wish to include an abstract, uncomment the lines below
% \begin{abstract}
% Abstract text
% \end{abstract}

%----------------------------------------------------------------------------------------
%	SECTION 0
%----------------------------------------------------------------------------------------
\section{Updates on previous goals}
\begin{itemize}
	\item Finish debugging
	\item Design and add unit testing
	\item Hold a meeting with Josh and Alex about the summer project
\end{itemize}
%----------------------------------------------------------------------------------------
%	SECTION 1
%----------------------------------------------------------------------------------------
\section{Progress up to now}
\subsection{Debugging progress}\label{debug}
Last week after meeting with Josh and Alex, we decided that the migrating work for the material properties will be done by myself while Alex put all the efforts on Google protocol buffers (See Sec.\ \ref{alex}). 

So the migrated material properties is with a lot of new interfaces that needs me to adapt
the other pieces of codes in the debugging process. The ideas are
\begin{itemize}
	\item to fix all the data structure to using map type containers;
	\item to fix all the return-type functions (interfaces) to return const values.
\end{itemize}

At the end, Alex's new material prooerties class has to keep these data structures and interfaces however it differs internally.

After the moving finished yesterday, the re-debugging has to be done for the data structure change and corresponding syntax change. Up to now, roughly half of the classes can go through compilation and re-debugging should be finished by the end of this week.

\subsection{Marissa}
We held a half-hour long meeting on Monday to clarify the methodology about how to
apply two-grid method. I suggest Marissa test the two-grid method in fixed-source problems with
diffusion and we'll see if it works after Marissa get test results today or tomorrow.

\subsection{Alex's progress and summer plan}\label{alex}
As mentioned in the Subsection\ \ref{debug},\ I eventually asked Alex not to do the moving
the MaterialProperties from dev branch to restart branch 
as I think the protocol-related way is the better way to do material-related work. Instead,
I suggested Alex directly work with Josh on protocol buffer and I rewrite the MaterialProperties
(for the purpose of restarting BART).

As Alex's style, I requested he staying in Josh's office whenever Josh's in office. It turns out that
this works super well in terms of efficiency (he's not a home-working person).

After the interactions between Josh and Alex during the past week, Josh came up a plan which
looks reasonable and detailed enough with work to be done and certain measures to see how's 
Alex's work at the end of the summer.
%----------------------------------------------------------------------------------------
%	SECTION 2
%----------------------------------------------------------------------------------------
%\section{Things you need from Rachel}


%----------------------------------------------------------------------------------------
%	SECTION 3
%----------------------------------------------------------------------------------------
\section{Goals/Things will be going on}
\begin{itemize}
	\item Finish re-debugging.
	\item Continue to add unit tests.
	\item Help Marissa get two-grid to work.
\end{itemize}

%----------------------------------------------------------------------------------------
%	SECTION 4
%----------------------------------------------------------------------------------------
%\section{Links to any related materials}
%Google protocol buffers is here: https://developers.google.com/protocol-buffers/


%----------------------------------------------------------------------------------------
%	BIBLIOGRAPHY
%----------------------------------------------------------------------------------------

%\bibliographystyle{apalike}
%
%\bibliography{sample}

%----------------------------------------------------------------------------------------


\end{document}