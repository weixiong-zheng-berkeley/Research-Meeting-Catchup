
%%%%%%%%%%%%%%%%%%%%%%%%%%%%%%%%%%%%%%%%%
% University/School Laboratory Report
% LaTeX Template
% Version 3.1 (25/3/14)
%
% This template has been downloaded from:
% http://www.LaTeXTemplates.com
%
% Original author:
% Linux and Unix Users Group at Virginia Tech Wiki 
% (https://vtluug.org/wiki/Example_LaTeX_chem_lab_report)
%
% License:
% CC BY-NC-SA 3.0 (http://creativecommons.org/licenses/by-nc-sa/3.0/)
%
%%%%%%%%%%%%%%%%%%%%%%%%%%%%%%%%%%%%%%%%%

%----------------------------------------------------------------------------------------
%	PACKAGES AND DOCUMENT CONFIGURATIONS
%----------------------------------------------------------------------------------------

\documentclass{article}
\usepackage[margin=1.25in]{geometry}
\usepackage{hyperref}
\usepackage[version=3]{mhchem} % Package for chemical equation typesetting
\usepackage{siunitx} % Provides the \SI{}{} and \si{} command for typesetting SI units
\usepackage{graphicx} % Required for the inclusion of images
\usepackage{natbib} % Required to change bibliography style to APA
\usepackage{amsmath} % Required for some math elements 

\setlength\parindent{1em} % Removes all indentation from paragraphs
\setlength{\parskip}{1em}
\renewcommand{\labelenumi}{\alph{enumi}.} % Make numbering in the enumerate environment by letter rather than number (e.g. section 6)

%\usepackage{times} % Uncomment to use the Times New Roman font

%----------------------------------------------------------------------------------------
%	DOCUMENT INFORMATION
%----------------------------------------------------------------------------------------

\title{Check-in} % Title

\author{Weixiong Zheng} % Author name

\date{\today} % Date for the report

\begin{document}

\maketitle % Insert the title, author and date
% If you wish to include an abstract, uncomment the lines below
% \begin{abstract}
% Abstract text
% \end{abstract}

%----------------------------------------------------------------------------------------
%	SECTION 0
%----------------------------------------------------------------------------------------
\section{Updates on previous goals}
\begin{itemize}
	\item Rewrite the {\tt BARTDriver}
	\item Plan for Alex's summer project
	\item Provide instructions/helps for Marissa
\end{itemize}
%----------------------------------------------------------------------------------------
%	SECTION 1
%----------------------------------------------------------------------------------------
\section{Progress up to now}
\subsection{Rewriting and {\tt BARTDriver<dim>}}
The entire rewriting process (except for the {\tt MaterialProperties}) is finished on June 13th. The main efforts were:
\begin{itemize}
	\item Adapting the migrating classes to the resource struct 
	{\tt FundamentalData<dim>}
	\item Adding the missing documentations in the rewriting
	\item Rewriting {\tt BARTDriver<dim>}
\end{itemize}

In the end, the {\tt BARTDriver} is performing functionalities included in five member 
functions:
\begin{itemize}
	\item {\tt MakeGrid}: call {\tt MeshGenerator<dim>::MakeGrid} to generate a mesh
	\item {\tt InitMatVec}: initialize deal.II and PETSc data structures (SparseMatrix and distributed vectors) contained in {\tt FundamentalData<dim>}
	\item {\tt DoIterations}: call {\tt Iterations<dim>::DoIterations} which internally call {\tt EigenBase<dim>::DoIterations} or {\tt MGBase<dim>::DoIterations} depending on the input parameters.
	\item {\tt OutputResults}: output moments to .pvtu files with newly added feature that we can also have keff in the file.
	\item {\tt DriveBART}: call the four equations above to do the whole calculations.
\end{itemize}

\subsection{Alex's summer plan}
I spent two meetings with Josh to talk about the material inputing. The main issue is 
current input of materials is blended with all other parameter inputs. Though it's 
possible to do large test problems using current input method, it's gonna be bulky. My
first idea about modifying this was to use .xml input to isolate the cross sections from
the parameter input. Josh, then propose to blend .xml things with {Google protocol buffers}, which provides an automatic way of generating parsing functions. The main complain about .xml from Josh was that we have to develop parsers,
which is problematic and hard to maintain.

So I think Alex's summer project is quite clear, which consists of the following parts:
\begin{itemize}
	\item Finish current moving things with current input method. This is still useful in
	terms of testing as many classes depend on MaterialProperties so we have to have
	one in working condition.
	\item Develop .xml format for the cross sections
	\item Develop script to generate Google protocol buffer string
	\item Correctly generate parsing function libraries based on Google protocol buffer
	\item Reimplement MaterialProperties using parsing functions generated by Google protocol buffers
	\item Provide complete unit testing for the re-implementations.
\end{itemize}

\subsection{Marissa}
Three long meetings were scheduled to help Marissa test and debug. What we ended up 
getting were:
\begin{itemize}
	\item One analytic solution for multigroup fixed-source problem in infinite medium.
	\item One analytic solution for multigroup eigenvalue problem in infinite medium.
\end{itemize}

It seems like the Gauss-Seidel iteration matches the analytic solution for the fixed-source
benchmark. But the power iteration still does not give correct solution after hours of
debugging. Yet, after hours of debugging, Marissa's code can generate rational values 
for keff with me fixing a hidden bug, still cannot give correct answers.

So, we changed the direction that she starts to develop NDA and set debugging aside 
for a little while. 
%----------------------------------------------------------------------------------------
%	SECTION 2
%----------------------------------------------------------------------------------------
%\section{Things you need from Rachel}


%----------------------------------------------------------------------------------------
%	SECTION 3
%----------------------------------------------------------------------------------------
\section{Goals/Things will be going on}
\begin{itemize}
	\item Finish debugging
	\item Design and add unit testing
	\item Hold a meeting with Josh and Alex about the summer project
\end{itemize}

%----------------------------------------------------------------------------------------
%	SECTION 4
%----------------------------------------------------------------------------------------
\section{Links to any related materials}
Google protocol buffers is here: https://developers.google.com/protocol-buffers/


%----------------------------------------------------------------------------------------
%	BIBLIOGRAPHY
%----------------------------------------------------------------------------------------

%\bibliographystyle{apalike}
%
%\bibliography{sample}

%----------------------------------------------------------------------------------------


\end{document}