%%%%%%%%%%%%%%%%%%%%%%%%%%%%%%%%%%%%%%%%%
% University/School Laboratory Report
% LaTeX Template
% Version 3.1 (25/3/14)
%
% This template has been downloaded from:
% http://www.LaTeXTemplates.com
%
% Original author:
% Linux and Unix Users Group at Virginia Tech Wiki 
% (https://vtluug.org/wiki/Example_LaTeX_chem_lab_report)
%
% License:
% CC BY-NC-SA 3.0 (http://creativecommons.org/licenses/by-nc-sa/3.0/)
%
%%%%%%%%%%%%%%%%%%%%%%%%%%%%%%%%%%%%%%%%%

%----------------------------------------------------------------------------------------
%	PACKAGES AND DOCUMENT CONFIGURATIONS
%----------------------------------------------------------------------------------------

\documentclass{article}
\usepackage{hyperref}
\usepackage[version=3]{mhchem} % Package for chemical equation typesetting
\usepackage{siunitx} % Provides the \SI{}{} and \si{} command for typesetting SI units
\usepackage{graphicx} % Required for the inclusion of images
\usepackage{natbib} % Required to change bibliography style to APA
\usepackage{amsmath} % Required for some math elements 

\setlength\parindent{1em} % Removes all indentation from paragraphs
\setlength{\parskip}{1em}
\renewcommand{\labelenumi}{\alph{enumi}.} % Make numbering in the enumerate environment by letter rather than number (e.g. section 6)

%\usepackage{times} % Uncomment to use the Times New Roman font

%----------------------------------------------------------------------------------------
%	DOCUMENT INFORMATION
%----------------------------------------------------------------------------------------

\title{Weekly Recap} % Title

\author{Weixiong Zheng} % Author name

\date{\today} % Date for the report

\begin{document}

\maketitle % Insert the title, author and date
% If you wish to include an abstract, uncomment the lines below
% \begin{abstract}
% Abstract text
% \end{abstract}

%----------------------------------------------------------------------------------------
%	SECTION 0
%----------------------------------------------------------------------------------------
\section{Updates on previous goals}
Previous goals was to start on unit test after finishing style guide.

%----------------------------------------------------------------------------------------
%	SECTION 1
%----------------------------------------------------------------------------------------
\section{Progress up to now}
Added modifications throughout the progress of starting unit test, e.g. how to break up long lines generally (not limited to parenthesis).

In terms of unit testing, I just finished up the first unit testing for {\tt AQBase<dim>}, an abstract class using Google Test and part of the functionalities from Google Mock. It was a good practice as I have to know Google Test/Mock as well as some {\tt deal.II} functionalities on parameter parsing.

%----------------------------------------------------------------------------------------
%	SECTION 2
%----------------------------------------------------------------------------------------
\section{Things you need from Rachel}
Not up to now.

%----------------------------------------------------------------------------------------
%	SECTION 3
%----------------------------------------------------------------------------------------
\section{Goals for the coming week}
Continue on unit testing. Meeting up with Josh if possible to see how we get the tests really to test. In the meantime, migrating the derived class of {\tt  AQBase<dim>} with adding unit tests.

%----------------------------------------------------------------------------------------
%	SECTION 4
%----------------------------------------------------------------------------------------
\section{Links to any related materials}
All the things not ready yet on BART upstream. But you can take a look by cloning my origin and see the doxygen files or see the html files:  \url{https://github.com/weixiong-zheng-berkeley/BART/tree/restart/html}.


%----------------------------------------------------------------------------------------
%	BIBLIOGRAPHY
%----------------------------------------------------------------------------------------

%\bibliographystyle{apalike}
%
%\bibliography{sample}

%----------------------------------------------------------------------------------------


\end{document}