%%%%%%%%%%%%%%%%%%%%%%%%%%%%%%%%%%%%%%%%%
% University/School Laboratory Report
% LaTeX Template
% Version 3.1 (25/3/14)
%
% This template has been downloaded from:
% http://www.LaTeXTemplates.com
%
% Original author:
% Linux and Unix Users Group at Virginia Tech Wiki 
% (https://vtluug.org/wiki/Example_LaTeX_chem_lab_report)
%
% License:
% CC BY-NC-SA 3.0 (http://creativecommons.org/licenses/by-nc-sa/3.0/)
%
%%%%%%%%%%%%%%%%%%%%%%%%%%%%%%%%%%%%%%%%%

%----------------------------------------------------------------------------------------
%	PACKAGES AND DOCUMENT CONFIGURATIONS
%----------------------------------------------------------------------------------------

\documentclass{article}
\usepackage[margin=1.25in]{geometry}
\usepackage{hyperref}
\usepackage[version=3]{mhchem} % Package for chemical equation typesetting
\usepackage{siunitx} % Provides the \SI{}{} and \si{} command for typesetting SI units
\usepackage{graphicx} % Required for the inclusion of images
\usepackage{natbib} % Required to change bibliography style to APA
\usepackage{amsmath} % Required for some math elements 

\setlength\parindent{1em} % Removes all indentation from paragraphs
\setlength{\parskip}{1em}
\renewcommand{\labelenumi}{\alph{enumi}.} % Make numbering in the enumerate environment by letter rather than number (e.g. section 6)

%\usepackage{times} % Uncomment to use the Times New Roman font

%----------------------------------------------------------------------------------------
%	DOCUMENT INFORMATION
%----------------------------------------------------------------------------------------

\title{Weekly Catchup} % Title

\author{Weixiong Zheng} % Author name

\date{\today} % Date for the report

\begin{document}

\maketitle % Insert the title, author and date
% If you wish to include an abstract, uncomment the lines below
% \begin{abstract}
% Abstract text
% \end{abstract}

%----------------------------------------------------------------------------------------
%	SECTION 0
%----------------------------------------------------------------------------------------
\section{Updates on previous goals}
Previous goal was to make MPI work with ctest.
%----------------------------------------------------------------------------------------
%	SECTION 1
%----------------------------------------------------------------------------------------
\section{Progress up to now}
From last check-in, examples on how to use CTest, including dummy demos and real unit 
tests, for unit testing are given. Yet, MPI was not necessary at that point. When adding unit 
test for MeshGenerator<dim>, however, MPI has to be involved.

Therefore, first part of the efforts was put on showing how MPI is used to work with unit 
test. A dummy demo without involving BART classes was given. Basically, writing a 
MPI-related unit test with CTest is to write a complete MPI program. As current unit test
is logging-based, a collective logging functionality was developed based on deal.II's test suite example with modifications.

The other part of efforts was to simplify how to initialize/finalize logging in unit testing, 
which is the foundation of the ctest with deal.II. Wrapper functions/struct's are provided for 
developers to call and will appear in next PR, which include both serial tests and MPI-based tests.

The last part was to show how CTest can be done with external input files. Though executables
can find input files specified by name (in the same directory as the test source file), CTest
would not be able to find it. First trial was to use C function getcwd to generate the path for that file so the filename is completed with its absolute path. It turned out unit test still failed. 
Suggestions from developers was to use macros to define the absolute path, which turned me 
to playing game with CMake for a while. It paid off at the end.

In parallel, Josh is trying to develop a unit test framework based on GTest. One benefit would be 
no extra configuration needs to be done with Travis CI and test coverage and they have been 
done with GTest before in BART. We will know the progress at tomorrow's BART meeting. As he
prefer GTest over CTest, we got an agreement of transferring CTest to GTest if he can get MPI
to work with GTest. This can be good as each test with CTest using MPI has to individually
initialize and finalize MPI, which can be time consuming considering we will fully test BART. Yet,
Josh claims GTest could work in a way s.t. MPI will only be initialized once and all tests 
requesting MPI will be executed during that time and then MPI will be finalized.

For Sam, an initial plan is to try to write a 1D code in a separate repo using deal.II. The plan was based on his familiarity with 1D formulation. It will also serve as a demo using 2 different finite element types which can be useful for BART and will be incorporated in Josh's project.

For Alex, we haven't discussed for real detail for both serial and MPI-based. He started installing deal.II from last Wednesday and I haven't heard back from him yet. I will talk with him about what we can do afterwards.

%----------------------------------------------------------------------------------------
%	SECTION 2
%----------------------------------------------------------------------------------------
\section{Things you need from Rachel}


%----------------------------------------------------------------------------------------
%	SECTION 3
%----------------------------------------------------------------------------------------
\section{Goals/Things will be going on}
Put instructions in style guide about how to do unit testing. Also, work with students to get everybody smoothly started.

%----------------------------------------------------------------------------------------
%	SECTION 4
%----------------------------------------------------------------------------------------
%\section{Links to any related materials}



%----------------------------------------------------------------------------------------
%	BIBLIOGRAPHY
%----------------------------------------------------------------------------------------

%\bibliographystyle{apalike}
%
%\bibliography{sample}

%----------------------------------------------------------------------------------------


\end{document}