%%%%%%%%%%%%%%%%%%%%%%%%%%%%%%%%%%%%%%%%%
% University/School Laboratory Report
% LaTeX Template
% Version 3.1 (25/3/14)
%
% This template has been downloaded from:
% http://www.LaTeXTemplates.com
%
% Original author:
% Linux and Unix Users Group at Virginia Tech Wiki 
% (https://vtluug.org/wiki/Example_LaTeX_chem_lab_report)
%
% License:
% CC BY-NC-SA 3.0 (http://creativecommons.org/licenses/by-nc-sa/3.0/)
%
%%%%%%%%%%%%%%%%%%%%%%%%%%%%%%%%%%%%%%%%%

%----------------------------------------------------------------------------------------
%	PACKAGES AND DOCUMENT CONFIGURATIONS
%----------------------------------------------------------------------------------------

\documentclass{article}
\usepackage{hyperref}
\usepackage[version=3]{mhchem} % Package for chemical equation typesetting
\usepackage{siunitx} % Provides the \SI{}{} and \si{} command for typesetting SI units
\usepackage{graphicx} % Required for the inclusion of images
\usepackage{natbib} % Required to change bibliography style to APA
\usepackage{amsmath} % Required for some math elements 

\setlength\parindent{1em} % Removes all indentation from paragraphs
\setlength{\parskip}{1em}
\renewcommand{\labelenumi}{\alph{enumi}.} % Make numbering in the enumerate environment by letter rather than number (e.g. section 6)

%\usepackage{times} % Uncomment to use the Times New Roman font

%----------------------------------------------------------------------------------------
%	DOCUMENT INFORMATION
%----------------------------------------------------------------------------------------

\title{Weekly Recap} % Title

\author{Weixiong Zheng} % Author name

\date{\today} % Date for the report

\begin{document}

\maketitle % Insert the title, author and date
% If you wish to include an abstract, uncomment the lines below
% \begin{abstract}
% Abstract text
% \end{abstract}

%----------------------------------------------------------------------------------------
%	SECTION 0
%----------------------------------------------------------------------------------------
\section{Updates on previous goals}
Last week's goal was to finish up documenting BART using doxygen. Up to last report,
I got familiar with doxygen for basic functionalities and documented some of the classes
in {\tt BART}.

%----------------------------------------------------------------------------------------
%	SECTION 1
%----------------------------------------------------------------------------------------
\section{Progress up to now}
It goes well and the only remaining work is the documentation of part of {\tt EquationBase<dim>}.

Fortunately, the past week has been great so far in terms of adapting myself to enjoying
doxygen. What I have gradually tried are:
\begin{itemize}
	\item Optimizing visual appearance of doxygen output.
	\item Adding hyper link for things like some specific documentation or articles illustrating formulations.
	\item Adding LaTeX formulas and symbols through the documentation.
\end{itemize}

For purpose of explaining which does what, it will be done, but we later could continuously optimize the
way we document, for better clarification and visualization.

%----------------------------------------------------------------------------------------
%	SECTION 2
%----------------------------------------------------------------------------------------
\section{Things you need from Rachel}
Not up to now about doxygen, as I am comfortable with it right now. Though I will utilize 
the group resource on github for unit test, I might still need help with it as sometimes talking
is a faster way for me to understand for things I couldn't get through by reading.

%----------------------------------------------------------------------------------------
%	SECTION 3
%----------------------------------------------------------------------------------------
\section{Goals for the coming week}
It would be the beginning of the unit test. We will see how far I can go. Hopefully, it
will be as enjoyable as doxygen.

%----------------------------------------------------------------------------------------
%	SECTION 4
%----------------------------------------------------------------------------------------
\section{Links to any related materials}
All the things not ready yet on BART upstream. But you can take a look by cloning my origin and see the doxygen files or see the html files:  \url{https://github.com/weixiong-zheng-berkeley/BART/tree/devel/html}.


%----------------------------------------------------------------------------------------
%	BIBLIOGRAPHY
%----------------------------------------------------------------------------------------

%\bibliographystyle{apalike}
%
%\bibliography{sample}

%----------------------------------------------------------------------------------------


\end{document}