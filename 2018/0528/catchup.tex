
%%%%%%%%%%%%%%%%%%%%%%%%%%%%%%%%%%%%%%%%%
% University/School Laboratory Report
% LaTeX Template
% Version 3.1 (25/3/14)
%
% This template has been downloaded from:
% http://www.LaTeXTemplates.com
%
% Original author:
% Linux and Unix Users Group at Virginia Tech Wiki 
% (https://vtluug.org/wiki/Example_LaTeX_chem_lab_report)
%
% License:
% CC BY-NC-SA 3.0 (http://creativecommons.org/licenses/by-nc-sa/3.0/)
%
%%%%%%%%%%%%%%%%%%%%%%%%%%%%%%%%%%%%%%%%%

%----------------------------------------------------------------------------------------
%	PACKAGES AND DOCUMENT CONFIGURATIONS
%----------------------------------------------------------------------------------------

\documentclass{article}
\usepackage[margin=1.25in]{geometry}
\usepackage{hyperref}
\usepackage[version=3]{mhchem} % Package for chemical equation typesetting
\usepackage{siunitx} % Provides the \SI{}{} and \si{} command for typesetting SI units
\usepackage{graphicx} % Required for the inclusion of images
\usepackage{natbib} % Required to change bibliography style to APA
\usepackage{amsmath} % Required for some math elements 

\setlength\parindent{1em} % Removes all indentation from paragraphs
\setlength{\parskip}{1em}
\renewcommand{\labelenumi}{\alph{enumi}.} % Make numbering in the enumerate environment by letter rather than number (e.g. section 6)

%\usepackage{times} % Uncomment to use the Times New Roman font

%----------------------------------------------------------------------------------------
%	DOCUMENT INFORMATION
%----------------------------------------------------------------------------------------

\title{Check-in} % Title

\author{Weixiong Zheng} % Author name

\date{\today} % Date for the report

\begin{document}

\maketitle % Insert the title, author and date
% If you wish to include an abstract, uncomment the lines below
% \begin{abstract}
% Abstract text
% \end{abstract}

%----------------------------------------------------------------------------------------
%	SECTION 0
%----------------------------------------------------------------------------------------
\section{Updates on previous goals}
\begin{itemize}
	\item Continue on rewriting based on using struct to store data.
	\item Get Alex started.
	\item Provide instructions/helps for Marissa
\end{itemize}
%----------------------------------------------------------------------------------------
%	SECTION 1
%----------------------------------------------------------------------------------------
\section{Progress up to now}
\subsection{Rewriting}
Up to now, the rewriting process is progressing smoothly. In short, the new structure is:
\begin{itemize}
	\item {\tt FundamentalData} holds 
	\begin{itemize}
		\item Matrices and vectors;
		\item Cross section data;
		\item Mesh;
		\item deal.II data structure that'll be used for assembly.
	\end{itemize}
	\item {\tt EquationBase} is remade close to a pure operator on the data structures held by {\tt FundamentalData}.
	\item Hierarchically, every level of iteration class has a member variable of iteration class that is one-level lower than current class, e.g. {\tt EigenBase} has a member variable of {\tt MGBase} instance and {\tt MGBase} has a member variable of {\tt IGBase} instance.
\end{itemize}

In the end, it leads to the fact that
\begin{itemize}
	\item All data structure which needs to be initialized in {\tt BARTDriver} but lives inside {\tt EquationBase} are now inside {\tt FundamentalData} (struct). EquationBase is cleaned to be a pure operator.
	\item Iteration classes are made to be operators as well. They operate on EquationBase instance.
\end{itemize}

\subsection{Alex's progress}
Alex just started with adding a resolution to a conflict between gmock and deal.II, which never existed until deal.II 9.0.

\subsection{Marissa}
Finished meeting with Marissa for her unclearness about power iterations. We are setting up meetings off-campus in Hayward starbucks for extra meetings.
%----------------------------------------------------------------------------------------
%	SECTION 2
%----------------------------------------------------------------------------------------
\section{Things you need from Rachel}


%----------------------------------------------------------------------------------------
%	SECTION 3
%----------------------------------------------------------------------------------------
\section{Goals/Things will be going on}
\begin{itemize}
	\item Finish the {\tt BARTDriver} as a closure to the rewriting (coding part)
	\item Add unit tests
	\begin{itemize}
		\item Iterations can be easily tested by gtest
		\item ctest can be used for EquationBase as MPI will be needed.
	\end{itemize}
\end{itemize}

%----------------------------------------------------------------------------------------
%	SECTION 4
%----------------------------------------------------------------------------------------
%\section{Links to any related materials}



%----------------------------------------------------------------------------------------
%	BIBLIOGRAPHY
%----------------------------------------------------------------------------------------

%\bibliographystyle{apalike}
%
%\bibliography{sample}

%----------------------------------------------------------------------------------------


\end{document}