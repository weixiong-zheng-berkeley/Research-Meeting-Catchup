
%%%%%%%%%%%%%%%%%%%%%%%%%%%%%%%%%%%%%%%%%
% University/School Laboratory Report
% LaTeX Template
% Version 3.1 (25/3/14)
%
% This template has been downloaded from:
% http://www.LaTeXTemplates.com
%
% Original author:
% Linux and Unix Users Group at Virginia Tech Wiki 
% (https://vtluug.org/wiki/Example_LaTeX_chem_lab_report)
%
% License:
% CC BY-NC-SA 3.0 (http://creativecommons.org/licenses/by-nc-sa/3.0/)
%
%%%%%%%%%%%%%%%%%%%%%%%%%%%%%%%%%%%%%%%%%

%----------------------------------------------------------------------------------------
%	PACKAGES AND DOCUMENT CONFIGURATIONS
%----------------------------------------------------------------------------------------

\documentclass{article}
\usepackage[margin=1.25in]{geometry}
\usepackage{hyperref}
\usepackage[version=3]{mhchem} % Package for chemical equation typesetting
\usepackage{siunitx} % Provides the \SI{}{} and \si{} command for typesetting SI units
\usepackage{graphicx} % Required for the inclusion of images
\usepackage{natbib} % Required to change bibliography style to APA
\usepackage{amsmath} % Required for some math elements 

\setlength\parindent{1em} % Removes all indentation from paragraphs
\setlength{\parskip}{1em}
\renewcommand{\labelenumi}{\alph{enumi}.} % Make numbering in the enumerate environment by letter rather than number (e.g. section 6)

%\usepackage{times} % Uncomment to use the Times New Roman font

%----------------------------------------------------------------------------------------
%	DOCUMENT INFORMATION
%----------------------------------------------------------------------------------------

\title{Check-in} % Title

\author{Weixiong Zheng} % Author name

\date{\today} % Date for the report

\begin{document}

\maketitle % Insert the title, author and date
% If you wish to include an abstract, uncomment the lines below
% \begin{abstract}
% Abstract text
% \end{abstract}

%----------------------------------------------------------------------------------------
%	SECTION 0
%----------------------------------------------------------------------------------------
\section{Updates on previous goals}
\begin{itemize}
	\item Discuss details of testing for all the class we have right now and separate the workload for 
	Josh and me.
	\item Integrate the new material class with the restart
	\item Start the research purpose development.
\end{itemize}
%----------------------------------------------------------------------------------------
%	SECTION 1
%----------------------------------------------------------------------------------------
\section{Progress up to now}
\subsection{Success and failure of MPI Testing}\label{debug}
Success of MPI since last meeting. A PR has been reviewed by Josh and merged for that.
In short, we could do MPI testing by giving a TearDown function to finalize the MPI.

I thought it was a whole success and so I continued adding testings with MPIs and things compile
fine. But when running, it failed. More than a week was spent on finding where the error
came from.

It was finally realized that the MPI failed for multiple tests because the file test\_main.cc
written by Josh by separating testing part of previous main.cc was ignored in the compilation
process (for instance, including any random header file never existed would not cause an 
compilation error).
By default
gtest\_main.cc was used. Unfortunately, the default main function from gtest
does not have proper mechanism creating and destroying parallel environment, causing
the errors we have now.

I was trying to resolving this and later involved Josh in.

\subsection{Research purpose development}
We started scheduling hours long meeting on coding with BART, specifically about the finite 
element parts in EquationBase. Fortunately, Josh understood the logics quickly with taking detailed notes. What I feel is Doxygen would not be super useful
in starting-period. In-person explanation is much easier for new developer to understand
the whole thing. 
So we planned to meet in the coming week for both answering his existing
questions and talking more about other parts we haven't touched. 

The goodness is Josh loves to learn what's going on under the hood, not just learn the
data structures/logics of BART, but the logical relationship between BART and deal.II to
improve his understandability. So while he goes back and digest on what we talked, I suggested
two basic yet useful tutorials on deal.II as he requested useful tutorial steps.


\subsection{New material class integration}
This part is in progress and in a ongoing PR by Josh. I will just assist Josh on this if there's
any issue resolving the Travis failure.

%----------------------------------------------------------------------------------------
%	SECTION 2
%----------------------------------------------------------------------------------------
%\section{Things you need from Rachel}


%----------------------------------------------------------------------------------------
%	SECTION 3
%----------------------------------------------------------------------------------------
\section{Goals/Things will be going on}
\begin{itemize}
	\item Fix the MPI test failure with Josh and continue testing
	\item Be a consultant for Josh to accelerate BART implementations
\end{itemize}

%----------------------------------------------------------------------------------------
%	SECTION 4
%----------------------------------------------------------------------------------------
%\section{Links to any related materials}
%Google protocol buffers is here: https://developers.google.com/protocol-buffers/


%----------------------------------------------------------------------------------------
%	BIBLIOGRAPHY
%----------------------------------------------------------------------------------------

%\bibliographystyle{apalike}
%
%\bibliography{sample}

%----------------------------------------------------------------------------------------


\end{document}