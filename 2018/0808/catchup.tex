
%%%%%%%%%%%%%%%%%%%%%%%%%%%%%%%%%%%%%%%%%
% University/School Laboratory Report
% LaTeX Template
% Version 3.1 (25/3/14)
%
% This template has been downloaded from:
% http://www.LaTeXTemplates.com
%
% Original author:
% Linux and Unix Users Group at Virginia Tech Wiki 
% (https://vtluug.org/wiki/Example_LaTeX_chem_lab_report)
%
% License:
% CC BY-NC-SA 3.0 (http://creativecommons.org/licenses/by-nc-sa/3.0/)
%
%%%%%%%%%%%%%%%%%%%%%%%%%%%%%%%%%%%%%%%%%

%----------------------------------------------------------------------------------------
%	PACKAGES AND DOCUMENT CONFIGURATIONS
%----------------------------------------------------------------------------------------

\documentclass{article}
\usepackage[margin=1.25in]{geometry}
\usepackage{hyperref}
\usepackage[version=3]{mhchem} % Package for chemical equation typesetting
\usepackage{siunitx} % Provides the \SI{}{} and \si{} command for typesetting SI units
\usepackage{graphicx} % Required for the inclusion of images
\usepackage{natbib} % Required to change bibliography style to APA
\usepackage{amsmath} % Required for some math elements 

\setlength\parindent{1em} % Removes all indentation from paragraphs
\setlength{\parskip}{1em}
\renewcommand{\labelenumi}{\alph{enumi}.} % Make numbering in the enumerate environment by letter rather than number (e.g. section 6)

%\usepackage{times} % Uncomment to use the Times New Roman font

%----------------------------------------------------------------------------------------
%	DOCUMENT INFORMATION
%----------------------------------------------------------------------------------------

\title{Check-in} % Title

\author{Weixiong Zheng} % Author name

\date{\today} % Date for the report

\begin{document}

\maketitle % Insert the title, author and date
% If you wish to include an abstract, uncomment the lines below
% \begin{abstract}
% Abstract text
% \end{abstract}

%----------------------------------------------------------------------------------------
%	SECTION 0
%----------------------------------------------------------------------------------------
\section{Updates on previous goals}
\begin{itemize}
	\item Get the runtime error done so BART will be at where we restart
	\item See if there's anything I can help for Marissa
\end{itemize}
%----------------------------------------------------------------------------------------
%	SECTION 1
%----------------------------------------------------------------------------------------
\section{Progress up to now}
\subsection{Debugging progress and testing on Aug 01}\label{debug}
From last meeting until last Friday, I was working on debugging on runtime error, which took much more time than what I expected. There are various types of things causing this. One of them is about templating. One of them, for instance, comes from using serial {\tt Triangulation<dim>} for 1D and distributed one {\tt parallel::distributed::Triangulation<dim>} for multi-D. This ends up making me separately template methods wherever triangulation shows up.

\subsection{Debugging progress and testing on Aug 08}
The main progress was after finishing the debugging, I've kept experimenting on MPI related data structure for gtest, which will extensively needed in later work. What we ended up getting is that a BART testing environment for easy use in MPI is created. Normally using GTest, we do
\begin{verbatim}
class TestClass : public ::testing::Test {...}
\end{verbatim}
In situations with MPI using the newly created environment in BART, we do
\begin{verbatim}
class TestClassMPI : public btest::BARTParallelEnvironment {...}
\end{verbatim}
The goodness is one does not need to provide a {\tt void TearDown()} to finalize MPI in the testing class. The {\tt btest::BARTParallelEnvironment} is implemented with MPI finalization functionality in its override of {::testing::Test::TearDown()}. Moreover, it provides a function to initialize MPI
\begin{verbatim}
class BARTParallelEnvironment : public ::testing::Test {
 public:
  void MPIInit () {
     char** argv;
     int argc = 0;
     int err = MPI_Init(&argc, &argv);
     ASSERT_FALSE(err);
  }
  
  void TearDown(){
    int err = MPI_Finalize();
    ASSERT_FALSE(err);
  }
};
\end{verbatim}

The resulting way of doing testing is super easy now with MPI. Beside that one does not need to provide TearDown, when overriding {\tt ::testing::Test::SetUp}, one only need to add one more line to initialize MPI compared to serial testing:
\begin{verbatim}
TestClassMPI::SetUp() {
  this->MPIInit();
  ... // whatever was planned to do
}
\end{verbatim}
\subsection{Marissa}
\subsubsection{Before Aug 01}
Multiple iterations were done to elaborate the thesis including formulation modification, data display, result interpretation etc.

\subsubsection{On Aug 08}
The main work was to direct Marissa doing literature review and filling in proper references in the thesis.

\subsection{What is planned next?}
Talking to Josh today for the work separation and introduce the MPI gtest environment usage.

%----------------------------------------------------------------------------------------
%	SECTION 2
%----------------------------------------------------------------------------------------
%\section{Things you need from Rachel}


%----------------------------------------------------------------------------------------
%	SECTION 3
%----------------------------------------------------------------------------------------
\section{Goals/Things will be going on}
\begin{itemize}
	\item Discuss details of testing for all the class we have right now and separate the workload for us two. It's too much for me to do alone. Besides, it will be a good practice for Josh.
	\item Integrate the new material class with the restart
	\item Start the research purpose development.
\end{itemize}

%----------------------------------------------------------------------------------------
%	SECTION 4
%----------------------------------------------------------------------------------------
%\section{Links to any related materials}
%Google protocol buffers is here: https://developers.google.com/protocol-buffers/


%----------------------------------------------------------------------------------------
%	BIBLIOGRAPHY
%----------------------------------------------------------------------------------------

%\bibliographystyle{apalike}
%
%\bibliography{sample}

%----------------------------------------------------------------------------------------


\end{document}