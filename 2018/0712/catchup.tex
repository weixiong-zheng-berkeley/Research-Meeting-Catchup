
%%%%%%%%%%%%%%%%%%%%%%%%%%%%%%%%%%%%%%%%%
% University/School Laboratory Report
% LaTeX Template
% Version 3.1 (25/3/14)
%
% This template has been downloaded from:
% http://www.LaTeXTemplates.com
%
% Original author:
% Linux and Unix Users Group at Virginia Tech Wiki 
% (https://vtluug.org/wiki/Example_LaTeX_chem_lab_report)
%
% License:
% CC BY-NC-SA 3.0 (http://creativecommons.org/licenses/by-nc-sa/3.0/)
%
%%%%%%%%%%%%%%%%%%%%%%%%%%%%%%%%%%%%%%%%%

%----------------------------------------------------------------------------------------
%	PACKAGES AND DOCUMENT CONFIGURATIONS
%----------------------------------------------------------------------------------------

\documentclass{article}
\usepackage[margin=1.25in]{geometry}
\usepackage{hyperref}
\usepackage[version=3]{mhchem} % Package for chemical equation typesetting
\usepackage{siunitx} % Provides the \SI{}{} and \si{} command for typesetting SI units
\usepackage{graphicx} % Required for the inclusion of images
\usepackage{natbib} % Required to change bibliography style to APA
\usepackage{amsmath} % Required for some math elements 

\setlength\parindent{1em} % Removes all indentation from paragraphs
\setlength{\parskip}{1em}
\renewcommand{\labelenumi}{\alph{enumi}.} % Make numbering in the enumerate environment by letter rather than number (e.g. section 6)

%\usepackage{times} % Uncomment to use the Times New Roman font

%----------------------------------------------------------------------------------------
%	DOCUMENT INFORMATION
%----------------------------------------------------------------------------------------

\title{Check-in} % Title

\author{Weixiong Zheng} % Author name

\date{\today} % Date for the report

\begin{document}

\maketitle % Insert the title, author and date
% If you wish to include an abstract, uncomment the lines below
% \begin{abstract}
% Abstract text
% \end{abstract}

%----------------------------------------------------------------------------------------
%	SECTION 0
%----------------------------------------------------------------------------------------
\section{Updates on previous goals}
\begin{itemize}
	\item Finish debugging
	\item Design and add unit testing
	\item Put efforts on gtest+MPI
	\item Help Marissa
\end{itemize}
%----------------------------------------------------------------------------------------
%	SECTION 1
%----------------------------------------------------------------------------------------
\section{Progress up to now}
\subsection{Debugging progress and testing}\label{debug}
During last week, the compilation errors were finally resolved. The final efforts was actually
fighting on one single linker error from the reimplementation of material properties due to 
lack of understanding {\tt inline}. What happened was I added {\tt inline} to small functions like
the interface functions while those functions, following Google Style's requirement were set
to be {\tt const} in the reimplementation. 

I am now fighting with runtime error and due to the restructuring. These will be eliminated and
unit testing will be set up.

\subsection{Gtest+MPI}
Part of the efforts was to understand what exactly we need to do to enable the MPI-gtest combo.
I talked to Josh and he had no idea. I finally got through this in terms of understanding. 

Basically what's happening is by default, gtest environment class {::testing::Environment} does 
not provide any functionality to initialize and terminate MPI. What we really need to do is 
to provide a child class and override the {\tt SetUp()} class to initialize MPI using {\tt MPI\_Init} and override {\tt TearDown()}
class to Finalize MPI using {\tt MPI\_Finalize} from STL.

The goodness is other than overriding functions from {::testing::Environment} everything will be almost the same.

The code snippet has been provided to Josh so he can include it when he separates the testing main function
from the BART main function for better understandability/readability. Hopefully, this will be reflected in the PR.

\subsection{Marissa}
Several meetings and online chatting were scheduled to debug the code. Two-grid part of the code, from today,
finally gives consistent result with NDA.

Through the brain damaging process, the difference between NDA and SAAF is reduced by changing to using
Gauss node

The other thing is Marissa wants to do a more realistic code. So some efforts were put on
getting familiar with my previous NJOY scripts and adapt them to generating the desirable 
cross sections. And we shall see if this works. In the meantime, I suggest Marissa use
moderator cross section from C5G7 and produce an artificial absorber by lowering
scattering cross sections by 20\% to be used in the shielding test. We will see how this works too.

\subsection{Alex's progress and summer plan}\label{alex}
As mentioned in the Subsection\ \ref{debug},\ I eventually asked Alex not to do the moving
the MaterialProperties from dev branch to restart branch 
as I think the protocol-related way is the better way to do material-related work. Instead,
I suggested Alex directly work with Josh on protocol buffer and I rewrite the MaterialProperties
(for the purpose of restarting BART).

As Alex's style, I requested he staying in Josh's office whenever Josh's in office. It turns out that
this works super well in terms of efficiency (he's not a home-working person).

After the interactions between Josh and Alex during the past week, Josh came up a plan which
looks reasonable and detailed enough with work to be done and certain measures to see how's 
Alex's work at the end of the summer.
%----------------------------------------------------------------------------------------
%	SECTION 2
%----------------------------------------------------------------------------------------
%\section{Things you need from Rachel}


%----------------------------------------------------------------------------------------
%	SECTION 3
%----------------------------------------------------------------------------------------
\section{Goals/Things will be going on}
\begin{itemize}
	\item Get the runtime error done so BART will be at where we restart
	\item Continue to add unit tests.
	\item Get gtest to work with MPI (with Josh).
	\item See if there's anything I can help for Marissa
\end{itemize}

%----------------------------------------------------------------------------------------
%	SECTION 4
%----------------------------------------------------------------------------------------
%\section{Links to any related materials}
%Google protocol buffers is here: https://developers.google.com/protocol-buffers/


%----------------------------------------------------------------------------------------
%	BIBLIOGRAPHY
%----------------------------------------------------------------------------------------

%\bibliographystyle{apalike}
%
%\bibliography{sample}

%----------------------------------------------------------------------------------------


\end{document}